% filepath: d:\21_school\T10D16.ID_239637-1-develop\10.tex
\documentclass[a4paper,11pt]{article}
\usepackage[utf8]{inputenc}
\usepackage[T2A]{fontenc}
\usepackage[russian]{babel}
\usepackage{amsmath}
\usepackage{hyperref}
\title{Отчёт по заданиям Quest 1 и Quest 2}
\author{Автоматизированный отчёт}
\date{\today}

\begin{document}
\maketitle

\section{Цель задачи}
Реализовать в рамках проекта функции библиотеки строк: \texttt{s21\_strlen} (Quest~1) и \texttt{s21\_strcmp} (Quest~2) и разработать модульные тесты для каждой функции. Обеспечить сборку тестовых программ с помощью Makefile (стадии \texttt{strlen\_tests} и \texttt{strcmp\_tests}) и сохранить исполняемые файлы в папке \texttt{build} под именами \texttt{Quest\_1} и \texttt{Quest\_2} соответственно.

\section{Описание шагов выполнения}
\begin{enumerate}
  \item Создан заголовочный файл \texttt{src/s21\_string.h} с объявлениями функций.
  \item Реализована функция \texttt{s21\_strlen} в \texttt{src/s21\_string.c} (уже выполнено в предыдущем шаге).
  \item Реализована функция \texttt{s21\_strcmp} в \texttt{src/s21\_string.c} с учётом простого безопасного поведения при \texttt{NULL} (обе указателя NULL --- 0; NULL меньше ненулевого).
  \item Добавлены модульные тесты в \texttt{src/s21\_string\_test.c}: \texttt{s21\_strlen\_test} и \texttt{s21\_strcmp\_test}. Каждый тест выводит: входные данные, результат вызова функции и итог проверки (SUCCESS/FAIL).
  \item Обновлён Makefile: добавлены цели \texttt{strlen\_tests} и \texttt{strcmp\_tests}, формирующие \texttt{build/Quest\_1} и \texttt{build/Quest\_2}.
\end{enumerate}

\section{Применённые методы и подходы}
\subsection{Реализация функций}
\begin{itemize}
  \item s21\_strlen: классический проход по строке с подсчётом символов до первого '\verb|\0|' и возвратом длины.
  \item s21\_strcmp: поэлементное побайтное сравнение (приведение к \texttt{unsigned char} для корректной арифметики). Поведение при \texttt{NULL} определено явно для безопасного тестирования.
\end{itemize}

\subsection{Тестирование}
Каждая тестовая функция содержит набор кейсов (нормальные, краевые и ненормальные). Для \texttt{s21\_strcmp} использованы примеры:
\begin{itemize}
  \item равные строки,
  \item лексикографически различающиеся строки,
  \item строка и её префикс,
  \item случаи с \texttt{NULL}.
\end{itemize}
По каждому случаю выводятся вход, выход и пометка SUCCESS или FAIL.

\section{Возможные сложности и способы их решения}
\begin{itemize}
  \item Передача \texttt{NULL} в стандартный \texttt{strcmp} --- UB. Для устойчивости тестов реализовано определённое поведение при \texttt{NULL}. При необходимости это поведение можно изменить, чтобы соответствовать строгому стандарту.
  \item Требования к средствам: запрещено использовать \texttt{string.h} и другие библиотеки --- реализовано только на языке C и с использованием разрешённых заголовков (\texttt{stdio.h}, \texttt{stdlib.h}).
  \item Стилевые и инструментальные проверки: в репозитории предусмотрены указания по использованию \texttt{clang-format} и инструментов проверки утечек памяти (Valgrind / leaks).
\end{itemize}

\section{Заключение и возможные улучшения}
Реализация выполнена минималистично и безопасно для тестового окружения. Возможные улучшения:
\begin{itemize}
  \item Согласование поведения при \texttt{NULL} с требованиями автотестов (если ожидается UB, убрать защиту).
  \item Добавление большего количества тестов (много длинные строки, нетипичные байты).
  \item Включение статических анализаторов и AddressSanitizer в Makefile для автоматической проверки памяти.
\end{itemize}

% filepath: d:\21_school\T10D16.ID_239637-1-develop\10.tex
% ...existing code...
\section{Quest 3: Реализация s21\_strcpy}

\subsection{Цель}
Реализовать функцию копирования строки s21\_strcpy и подготовить для неё модульные тесты (нормальные, краевые и ненормальные случаи). Обеспечить сборку тестовой версии через Makefile (стадия \texttt{strcpy\_tests}), исполняемый файл сохранить как \texttt{build/Quest\_3}.

\subsection{Шаги выполнения}
\begin{enumerate}
  \item Добавлено объявление \texttt{char *s21\_strcpy(char *dest, const char *src);} в \texttt{src/s21\_string.h}.
  \item Реализована функция в \texttt{src/s21\_string.c}: побайтное копирование символов из \texttt{src} в \texttt{dest} с копированием завершающего нуля.
  \item Добавлен тест \texttt{s21\_strcpy\_test} в \texttt{src/s21\_string\_test.c}, проверяющий:
    \begin{itemize}
      \item обычное копирование (\texttt{"sample"});
      \item копирование пустой строки;
      \item поведение при \texttt{src == NULL} (реализация записывает пустую строку в \texttt{dest}).
    \end{itemize}
  \item Обновлён Makefile — добавлена цель \texttt{strcpy\_tests}, собирающая \texttt{build/Quest\_3}.
\end{enumerate}

\subsection{Принятые подходы и обоснование}
Реализация использует простой цикл копирования с присваиванием через выражение \texttt{(*d++ = *s++)}. Для устойчивости тестовой среды определено поведение при NULL: если \texttt{dest == NULL} функция возвращает \texttt{NULL}; если \texttt{src == NULL} — функция записывает пустую строку в \texttt{dest} и возвращает \texttt{dest}. Такое поведение позволяет безопасно проверять функцию в тестах без возникновения неопределённого поведения.

\subsection{Возможные сложности и их решения}
\begin{itemize}
  \item Стандартное поведение при передаче \texttt{NULL} в \texttt{strcpy} --- неопределено. Для детерминированности тестов введено явно описанное поведение при \texttt{NULL}.
  \item Требование не использовать \texttt{string.h} соблюдено: все операции реализованы на уровне языка C.
  \item Проверки на утечки памяти и стиль кода остаются релевантными; тесты не используют динамическое выделение, поэтому утечек не возникает.
\end{itemize}

\subsection{Выводы и улучшения}
Реализация компактна и безопасна для модульного тестирования. В дальнейшем можно:
\begin{itemize}
  \item добавить тесты для перекрывающихся областей памяти (но поведение стандартной strcpy для перекрытия не определено — рекомендовать использовать \texttt{memmove} в реальных задачах);
  \item расширить набор тестов (очень длинные строки, байты >127);
  \item при необходимости изменить поведение при \texttt{NULL}, чтобы соответствовать строго стандартному поведению (UB).
\end{itemize}

% filepath: d:\21_school\T10D16.ID_239637-1-develop\10.tex
% ...existing code...
\section{Quest 4: Реализация s21\_strcat}

\subsection{Цель}
Реализовать функцию конкатенации строк s21\_strcat и подготовить модульные тесты для проверки нормальных, краевых и ненормальных случаев. Добавить стадию сборки \texttt{strcat\_tests}, формирующую \texttt{build/Quest\_4}.

\subsection{Шаги выполнения}
\begin{enumerate}
  \item Добавлено объявление \texttt{char *s21\_strcat(char *dest, const char *src);} в \texttt{src/s21\_string.h}.
  \item Реализована функция в \texttt{src/s21\_string.c}: поиск конца \texttt{dest} и посимвольное копирование \texttt{src}, включая завершающий ноль.
  \item Добавлен тест \texttt{s21\_strcat\_test} в \texttt{src/s21\_string\_test.c}, проверяющий:
    \begin{itemize}
      \item нормальную конкатенацию;
      \item конкатенацию в пустой буфер;
      \item конкатенацию с пустой \texttt{src};
      \item поведение при \texttt{src == NULL} (ничего не меняется);
      \item поведение при \texttt{dest == NULL} (функция возвращает \texttt{NULL}).
    \end{itemize}
  \item Обновлён Makefile — добавлена цель \texttt{strcat\_tests}, собирающая \texttt{build/Quest\_4}.
\end{enumerate}

\subsection{Методы и замечания}
Функция реализована на уровне языка C без использования \texttt{string.h}. Для устойчивости тестов определено поведение при \texttt{NULL} (возврат \texttt{NULL} при \texttt{dest == NULL}, при \texttt{src == NULL} ничего не меняется), что делает тестирование детерминированным.

\subsection{Возможные улучшения}
Добавить дополнительные кейсы (переполнение буфера, длинные строки), включить проверки с AddressSanitizer/Valgrind и расширить набор проверок на утечки и стиль.


% filepath: d:\21_school\T10D16.ID_239637-1-develop\10.tex
% ...existing code...
\section{Quest 5: Реализация s21\_strchr}

\subsection{Цель}
Реализовать функцию поиска первого вхождения символа в строке \texttt{s21\_strchr} и подготовить модульные тесты для проверки нормальных, краевых и ненормальных случаев. Добавить стадию сборки \texttt{strchr\_tests}, формирующую \texttt{build/Quest\_5}.

\subsection{Шаги выполнения}
\begin{enumerate}
  \item Добавлено объявление \texttt{char *s21\_strchr(const char *s, int c);} в \texttt{src/s21\_string.h}.
  \item Реализована функция в \texttt{src/s21\_string.c}: проход по строке с поиском первого совпадения, проверка завершающего NUL, безопасная обработка \texttt{NULL}.
  \item Добавлен тест \texttt{s21\_strchr\_test} в \texttt{src/s21\_string\_test.c}, проверяющий:
    \begin{itemize}
      \item поиск существующего символа;
      \item отсутствие символа в строке;
      \item поиск завершающего символа '\textbackslash0' (возврат указателя на конец строки);
      \item поведение при \texttt{s == NULL}.
    \end{itemize}
  \item Обновлён Makefile — добавлена цель \texttt{strchr\_tests}, собирающая \texttt{build/Quest\_5}.
\end{enumerate}

\subsection{Методы и замечания}
Функция реализована без использования \texttt{string.h}, при этом поведение при \texttt{NULL} явно определено (возврат \texttt{NULL}) для детерминированности тестов. Тесты выводят входные данные, полученный результат и пометку SUCCESS/FAIL.

\subsection{Возможные улучшения}
Добавить дополнительные кейсы (длинные строки, нечётные байтовые значения), включить автоматические проверки на переполнение буфера и статический анализ.

% ...existing code...

% filepath: d:\21_school\T10D16.ID_239637-1-develop\10.tex
% ...existing code...
\section{Quest 6: Реализация s21\_strstr}

\subsection{Цель}
Реализовать функцию поиска подстроки \texttt{s21\_strstr} и подготовить модульные тесты для проверки нормальных, краевых и ненормальных случаев. Добавить стадию сборки \texttt{strstr\_tests}, формирующую \texttt{build/Quest\_6}.

\subsection{Шаги выполнения}
\begin{enumerate}
  \item Добавлено объявление \texttt{char *s21\_strstr(const char *haystack, const char *needle);} в \texttt{src/s21\_string.h}.
  \item Реализована функция в \texttt{src/s21\_string.c}: наивный поиск подстроки с проверкой завершающего нуля и безопасной обработкой \texttt{NULL}.
  \item Добавлен тест \texttt{s21\_strstr\_test} в \texttt{src/s21\_string\_test.c}, проверяющий:
    \begin{itemize}
      \item нахождение существующей подстроки;
      \item отсутствие подстроки;
      \item пустой needle (возврат указателя на haystack);
      \item needle длиннее haystack;
      \item поведение при \texttt{NULL}-указателях.
    \end{itemize}
  \item Обновлён Makefile — добавлена цель \texttt{strstr\_tests}, собирающая \texttt{build/Quest\_6}.
\end{enumerate}

\subsection{Методы и замечания}
Реализация выполнена без использования \texttt{string.h}, используется наивный алгоритм поиска подстроки (два вложенных цикла). Для устойчивости тестовой среды поведение при \texttt{NULL} явно определено (возврат \texttt{NULL}). Тесты выводят входные данные, результат и пометку SUCCESS/FAIL.

\subsection{Возможные улучшения}
При необходимости заменить на более эффективный алгоритм (KMP) и добавить дополнительные тесты (длинные строки, разные байтовые значения), а также интегрировать проверки адресной/статической корректности в Makefile.


% filepath: d:\21_school\T10D16.ID_239637-1-develop\9.tex
\documentclass[a4paper,11pt]{article}
\usepackage[utf8]{inputenc}
\usepackage[T2A]{fontenc}
\usepackage[russian]{babel}
\usepackage{amsmath}
\title{Отчёт по заданию Quest 7: Реализация s21\_strtok}
\author{Автоматизированный отчёт}
\date{\today}

\begin{document}
\maketitle

\section{Цель задачи}
Реализовать функцию токенизации строк \texttt{s21\_strtok} и подготовить модульные тесты для проверки её работы в нормальных, краевых и ненормальных случаях. Обеспечить сборку тестовой программы через Makefile (стадия \texttt{strtok\_tests}) и сохранить исполняемый файл как \texttt{build/Quest\_7}.

\section{Шаги выполнения}
\begin{enumerate}
  \item Добавлено объявление функции \texttt{char *s21\_strtok(char *str, const char *delim);} в \texttt{src/s21\_string.h}.
  \item Реализована функция в \texttt{src/s21\_string.c} с использованием статического состояния для продолжения разбиения при повторных вызовах с \texttt{str == NULL}.
  \item Добавлен тест \texttt{s21\_strtok\_test} в \texttt{src/s21\_string\_test.c}, включающий несколько сценариев: нормальная токенизация, пробельные разделители, строка, содержащая только разделители, и проверка продолжения при передаче \texttt{NULL}.
  \item Обновлён Makefile: добавлена цель \texttt{strtok\_tests}, формирующая \texttt{build/Quest\_7}.
\end{enumerate}

\section{Применённые методы}
Реализация основана на классическом подходе \texttt{strtok}:
\begin{itemize}
  \item при начале разбора сохраняется текущая позиция в статическом указателе;
  \item при каждом вызове пропускаются ведущие разделители;
  \item найденный токен завершается записью нулевого символа, сохраняется позиция для следующего вызова;
  \item предусмотрена проверка на корректную работу при \texttt{NULL} в качестве входной строки или разделителей.
\end{itemize}

\section{Тестирование}
Тесты выводят входные данные, разделитель, найденные токены и итог проверки (SUCCESS/FAIL). Набор тестов охватывает:
\begin{itemize}
  \item стандартную CSV-подобную строку;
  \item строку с пробельными разделителями;
  \item строку, состоящую только из разделителей;
  \item последовательные вызовы с \texttt{NULL} для проверки сохранения состояния.
\end{itemize}

\section{Возможные улучшения}
\begin{itemize}
  \item покрыть дополнительные кейсы (многосимвольные разделители, UTF-8 байтовые особенности);
  \item добавить тесты на потокобезопасность (текущее поведение не является потокобезопасным, как и стандартный \texttt{strtok});
  \item интегрировать проверки статическим анализатором и AddressSanitizer в Makefile.
\end{itemize}

% filepath: d:\21_school\T10D16.ID_239637-1-develop\10.tex
% ...existing code...
\section{Quest 8: Реализация text\_processor (-w)}

\subsection{Цель}
Реализовать утилиту форматирования текста по заданной ширине (ключ \texttt{-w}) и предоставить краткий отчёт. Программа должна принимать ширину и строку текста из stdin и выводить форматированный по ширине текст. Собрать исполняемый файл \texttt{build/Quest\_8} через цель \texttt{text\_processor} в Makefile.

\subsection{Шаги выполнения}
\begin{enumerate}
  \item Добавлен файл \texttt{src/text\_processor.c}, принимающий ключ \texttt{-w}. При любых других ключах выводит \texttt{n/a}.
  \item Вход: целое число (ширина строки), затем строка текста (до конца строки). 
  \item Алгоритм: жадное заполнение строк, равномерное распределение пробелов между словами для выравнивания (для строк, отличных от последней), разбиение слишком длинных слов с помощью дефиса \texttt{-} по ширине (фрагменты длиной \texttt{width-1} плюс дефис).
  \item Обновлён Makefile: добавлена цель \texttt{text\_processor}, формирующая \texttt{build/Quest\_8}.
\end{enumerate}

\subsection{Особенности реализации}
\begin{itemize}
  \item Использованы только \texttt{stdio.h} и \texttt{stdlib.h}.
  \item Последняя строка выводится без завершающего перевода строки в соответствии с условием.
  \item Простой наивный алгоритм разбиения и справа-левая равномерная вставка пробелов реализованы для читабельности и детерминированности.
\end{itemize}

\subsection{Возможные улучшения}
\begin{itemize}
  \item Больше тестов (многократные пробелы, табуляции, Unicode/UTF-8);
  \item Улучшенная гибкая расстановка переносов (правила языка) вместо простого фиксированного разбиения;
  \item Интеграция с AddressSanitizer и статическим анализом для автоматической проверки.
\end{itemize}

\end{document}